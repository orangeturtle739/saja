\documentclass[a4paper]{article}

\usepackage[english]{babel}
\usepackage[utf8x]{inputenc}
\usepackage{amsmath}
\usepackage{amssymb}
\usepackage{graphicx}
\usepackage{booktabs}
\usepackage{placeins}
\usepackage{listings, lstautogobble}
\lstset{basicstyle=\ttfamily}

\title{Final Project Charter}
\author{Alex Libman (asl237), Jacob Glueck (jng55), \\Amit Mizrahi (am2269), Siddant Basnet (sb846)}
\date{}

\begin{document}
\maketitle

\section{Meeting Plan}
We plan to meet regularly on Tuesday nights at 2000 hours, and Saturdays at 1900 hours. 

\section{Proposal}
\subsection{Vision}
An extensible intelligent text-based assistant for creating and executing workflows.

\subsection{Key Features}
\begin{itemize}
	\item A domain specific language for processing user input and specifying the function to be performed
	\item A context-based system for grouping related tasks
	\item Ability to extend the current context interactively (defining new commands)
	\item Support for invoking arbitrary programs on the host machine within tasks
	\item Support for remote task execution on a server
\end{itemize}

\subsection{Description}
Our program is an extensible text based artificial intelligence system like Siri or Google Assistant. Interaction occurs within a \textit{context} which contains a list of \textit{rules}. Each rule contains two functions in our domain specific language: a function which takes a string and returns a boolean indicating whether to accept the string, and a function which takes a string and returns the processed output. The latter can have side effects and can invoke any arbitrary program on the host machine. Furthermore, the processing function may modify the rule list for this context or any other context, allowing the user to extend the functionality of the system.

Contexts form a tree structure, and the rules may change the current context to any other context within the tree. The root node of the tree is the \textit{top level} context. The top level context will come with standard functions for creating and modifying contexts.

\begin{lstlisting}
> create math // creates new math context
> switch math // enter math context 
math > rule "2 + 2 -> true" "4" 
// accept all strings of the format "2 + 2", output 4
math > 2 + 2
math: 4
// compute 2 + 2
math > "int: #1 + int: #2 -> true" "#1 + #2"
// defines a more general function which takes any string with 
// 2 integers being added together
math > 5 + 9
math: 14
// our new function computes 5 + 9
math > up
// go up a context
> "math expression: #1 -> true" "math(#1)"
// define a function which takes an argument and evaluates it in
// the math context
> math expression: 1 + 1
2
// evaluate a math expression in the top level
 
\end{lstlisting}

\end{document}